% Created 2021-11-03 mié 11:39
% Intended LaTeX compiler: lualatex
\documentclass[table]{scrartcl}
\usepackage[left=1.5cm,right=1.5cm,bottom=2.5cm,letterpaper]{geometry}
\usepackage[spanish, es-nodecimaldot, es-tabla]{babel}
\usepackage[utf8]{inputenc}
\usepackage{blindtext}
\usepackage{multicol}
\usepackage{subfigure}
\usepackage[most]{tcolorbox}
\usepackage{etoolbox}
\usepackage{minted}
\usepackage{hyperref}
% \usepackage[table,xcdraw]{xcolor}
\usepackage{longtable}
\usepackage{multirow}
\usepackage[default]{comfortaa}
\usepackage[T1]{fontenc}

\usepackage{caption}
\usepackage{breqn}
\usemintedstyle{emacs}
\usepackage[ruled,vlined]{algorithm2e}
\newenvironment{code}{\captionsetup{type=listing}}{}
\definecolor{custom}{HTML}{F8F8F8}
\setminted{frame=lines,breaklines=true,bgcolor=custom,fontsize=\scriptsize,linenos}
\renewcommand\listoflistingscaption{Índice de \listingscaption\@s}
\renewcommand{\listingscaption}{Código}
\BeforeBeginEnvironment{listing}{\begin{code}}
  \AfterEndEnvironment{listing}{\end{code}}
\BeforeBeginEnvironment{minted}{\begin{code}}
  \AfterEndEnvironment{minted}{\end{code}}
\author{Monsalvo Bolaños Melissa Monserrat y Romero Andrade Cristian}
\date{\today}
\title{Practica 4}
\hypersetup{
  pdfauthor={Monsalvo Bolaños Melissa Monserrat y Romero Andrade Cristian},
  pdftitle={Practica 3},
  pdfkeywords={},
  pdfsubject={},
  pdfcreator={Emacs 27.2 (Org mode 9.6)},
  pdflang={English}}

\newcommand{\tituloTrabajo}{Practica No. 4\\Construcción de Máquinas de
  estados Usando Memorias Direccionamiento
  Entrada - Estado}
\newcommand{\fechaEntrega}{07 de octubre de 2021}

\usepackage{subfiles}
\usepackage[backend=biber,style=apa]{biblatex}
\addbibresource{../bib.bib}

\begin{document}
\begin{titlepage}
  \centering

    {\scshape{\Huge Facultad de Ingeniería\par{}}}\vspace{0.25cm}

    \includegraphics[width=0.25\textwidth]{../img_common/unam_logo}\vspace{0.5cm}

    {\scshape{\Large Lab. Organización y Arquitectura de Computadoras\par{}}}\vfill{}


    {\huge \textbf{\tituloTrabajo{}}}\vfill{}


    {\Large
      Alumnos
      \begin{itemize}

        \item Monsalvo Bolaños Melissa Monserrat

        \item Romero Andrade Cristian
      \end{itemize}
    }\vfill{}

      {\large Grupo: 01\par{}}\vfill{}

    {\large Profesor\\Ing.~Adrian Ulises Mercado Martinez}\vfill{}
    \vfil{}
    {\large Semestre\\\textbf{2022--1}}
    \vfill{}
    {\large Fecha de Entrega\\\fechaEntrega}
    \vfill{}
    \includegraphics[width=0.1\textwidth]{../img_common/inge_logo}

\end{titlepage}

\maketitle{}
\tableofcontents{}
\section{Introducción}
\label{sec:org6bd500f}
El direccionamiento entrada-estado se restringe a cartas ASM con una sola
entrada por estado. Una nueva porción de la palabra de memoria contiene
una representación binaria de la entrada a probar en cada estado, esta parte
es llamada “PRUEBA”. Con esta representación binaria un selector de entrada
elige una de las variables de entrada. La parte de liga tiene dos estados
siguientes, encogiéndose uno por el selector de liga, en base a la entrada
seleccionada por la parte de prueba. Si el valor de la entrada seleccionada
por el selector de entradas es igual a cero, entonces el selector de liga elegirá
la liga falsa, en caso contrario se seleccionará la liga verdadera.
Este método tiene grandes ventajas como el ahorro de memoria, que
cuenta con pocos elementos de hardware y que representa un sistema muy
versátil.
\begin{figure}[htbp]
  \centering
  \includegraphics[width=0.6\textwidth]{./img/1.png}
  \caption{Direccionamiento Entrada-Estado}
\end{figure}
\section{Objetivo}
\label{sec:org8bfa7f0}
Familiarizar al alumno en el conocimiento de construcción de máquinas de
estados usando direccionamiento de memorias con el método de
direccionamiento entrada--estado.
\section{Desarrollo}
\label{sec:orgac7043c}
Empezamos analizando siguiente figura, donde esta es una carta ASM
donde secuencialmente otorgamos los valores binarios de los estados y para
los estados de prueba
\begin{figure}[htbp]
  \centering
  \includegraphics[width=0.6\textwidth]{./img/2.png}
  \caption{\label{fig:2}Carta ASM}
\end{figure}
\\
\begin{center}
    \captionof{table}{Valores binarios a estados}\label{tab:1}
  \begin{tabular}{rl}
    \multicolumn{2}{c}{\cellcolor[HTML]{EA4335}{\color[HTML]{FFFFFF} \textbf{Entradas}}} \\
    ESTA & 000 \\
    ESTB & 001 \\
    ESTC & 010 \\
    ESTD & 011 \\
    ESTE & 100 \\
    ESTF & 101 \\
    ESTG & 110
\end{tabular}
\end{center}
\begin{center}
  \captionof{table}{Valores binarios de las entradas}\label{tab:2}
\begin{tabular}{rl}
\multicolumn{2}{c}{\cellcolor[HTML]{EA4335}{\color[HTML]{FFFFFF} \textbf{Prueba}}} \\
x & 000 \\
y & 001 \\
z & 010 \\
w & 011 \\
aux & 100
\end{tabular}
\end{center}
Una vez especificados nuestros estados y entradas, resolvemos la tabla de
verdad.
\begin{center}
  \captionof{table}{Tabla de verdad obtenida}\label{tab:3}
  \scriptsize
  \begin{longtable}{|ccc|ccccccccccccccccccc|}
\hline
\rowcolor[HTML]{CC4125}
\multicolumn{3}{|c|}{\cellcolor[HTML]{CC4125}{\color[HTML]{FFFFFF} \textbf{Dirección de memoria}}} & \multicolumn{19}{c|}{\cellcolor[HTML]{CC4125}Contenido de la memoria} \\ \hline
\rowcolor[HTML]{E06666}
\multicolumn{3}{|c|}{\cellcolor[HTML]{E06666}Estado Presente} & \multicolumn{3}{c|}{\cellcolor[HTML]{E06666}} & \multicolumn{3}{c|}{\cellcolor[HTML]{E06666}} & \multicolumn{3}{c|}{\cellcolor[HTML]{E06666}} & \multicolumn{5}{c|}{\cellcolor[HTML]{E06666}Salidas Falsas} & \multicolumn{5}{c|}{\cellcolor[HTML]{E06666}Salidas Verdaderas} \\ \cline{1-3} \cline{13-22}
\rowcolor[HTML]{F6B26B}
\multicolumn{1}{|c|}{\cellcolor[HTML]{F6B26B}Q2} & \multicolumn{1}{c|}{\cellcolor[HTML]{F6B26B}Q1} & Q0 & \multicolumn{3}{c|}{\multirow{-2}{*}{\cellcolor[HTML]{E06666}Prueba}} & \multicolumn{3}{c|}{\multirow{-2}{*}{\cellcolor[HTML]{E06666}Liga Falsa}} & \multicolumn{3}{c|}{\multirow{-2}{*}{\cellcolor[HTML]{E06666}Liga Verdadera}} & \multicolumn{1}{c|}{\cellcolor[HTML]{F6B26B}S5} & \multicolumn{1}{c|}{\cellcolor[HTML]{F6B26B}S3} & \multicolumn{1}{c|}{\cellcolor[HTML]{F6B26B}S2} & \multicolumn{1}{c|}{\cellcolor[HTML]{F6B26B}S1} & \multicolumn{1}{c|}{\cellcolor[HTML]{F6B26B}S0} & \multicolumn{1}{c|}{\cellcolor[HTML]{F6B26B}S5} & \multicolumn{1}{c|}{\cellcolor[HTML]{F6B26B}S3} & \multicolumn{1}{c|}{\cellcolor[HTML]{F6B26B}S2} & \multicolumn{1}{c|}{\cellcolor[HTML]{F6B26B}S1} & S0 \\ \hline
\rowcolor[HTML]{FFE599}
\multicolumn{1}{|c|}{\cellcolor[HTML]{FFE599}0} & \multicolumn{1}{c|}{\cellcolor[HTML]{FFE599}0} & 0 & \multicolumn{1}{c|}{\cellcolor[HTML]{FFE599}1} & \multicolumn{1}{c|}{\cellcolor[HTML]{FFE599}0} & \multicolumn{1}{c|}{\cellcolor[HTML]{FFE599}0} & \multicolumn{1}{c|}{\cellcolor[HTML]{FFE599}0} & \multicolumn{1}{c|}{\cellcolor[HTML]{FFE599}0} & \multicolumn{1}{c|}{\cellcolor[HTML]{FFE599}1} & \multicolumn{1}{c|}{\cellcolor[HTML]{FFE599}0} & \multicolumn{1}{c|}{\cellcolor[HTML]{FFE599}0} & \multicolumn{1}{c|}{\cellcolor[HTML]{FFE599}1} & \multicolumn{1}{c|}{\cellcolor[HTML]{FFE599}0} & \multicolumn{1}{c|}{\cellcolor[HTML]{FFE599}0} & \multicolumn{1}{c|}{\cellcolor[HTML]{FFE599}0} & \multicolumn{1}{c|}{\cellcolor[HTML]{FFE599}0} & \multicolumn{1}{c|}{\cellcolor[HTML]{FFE599}0} & \multicolumn{1}{c|}{\cellcolor[HTML]{FFE599}0} & \multicolumn{1}{c|}{\cellcolor[HTML]{FFE599}0} & \multicolumn{1}{c|}{\cellcolor[HTML]{FFE599}0} & \multicolumn{1}{c|}{\cellcolor[HTML]{FFE599}0} & 0 \\ \hline
\rowcolor[HTML]{B6D7A8}
\multicolumn{1}{|c|}{\cellcolor[HTML]{B6D7A8}0} & \multicolumn{1}{c|}{\cellcolor[HTML]{B6D7A8}0} & 1 & \multicolumn{1}{c|}{\cellcolor[HTML]{B6D7A8}0} & \multicolumn{1}{c|}{\cellcolor[HTML]{B6D7A8}1} & \multicolumn{1}{c|}{\cellcolor[HTML]{B6D7A8}0} & \multicolumn{1}{c|}{\cellcolor[HTML]{B6D7A8}0} & \multicolumn{1}{c|}{\cellcolor[HTML]{B6D7A8}1} & \multicolumn{1}{c|}{\cellcolor[HTML]{B6D7A8}0} & \multicolumn{1}{c|}{\cellcolor[HTML]{B6D7A8}1} & \multicolumn{1}{c|}{\cellcolor[HTML]{B6D7A8}0} & \multicolumn{1}{c|}{\cellcolor[HTML]{B6D7A8}0} & \multicolumn{1}{c|}{\cellcolor[HTML]{B6D7A8}0} & \multicolumn{1}{c|}{\cellcolor[HTML]{B6D7A8}0} & \multicolumn{1}{c|}{\cellcolor[HTML]{B6D7A8}0} & \multicolumn{1}{c|}{\cellcolor[HTML]{B6D7A8}1} & \multicolumn{1}{c|}{\cellcolor[HTML]{B6D7A8}0} & \multicolumn{1}{c|}{\cellcolor[HTML]{B6D7A8}0} & \multicolumn{1}{c|}{\cellcolor[HTML]{B6D7A8}0} & \multicolumn{1}{c|}{\cellcolor[HTML]{B6D7A8}0} & \multicolumn{1}{c|}{\cellcolor[HTML]{B6D7A8}1} & 0 \\ \hline
\rowcolor[HTML]{A2C4C9}
\multicolumn{1}{|c|}{\cellcolor[HTML]{A2C4C9}0} & \multicolumn{1}{c|}{\cellcolor[HTML]{A2C4C9}1} & 0 & \multicolumn{1}{c|}{\cellcolor[HTML]{A2C4C9}1} & \multicolumn{1}{c|}{\cellcolor[HTML]{A2C4C9}0} & \multicolumn{1}{c|}{\cellcolor[HTML]{A2C4C9}0} & \multicolumn{1}{c|}{\cellcolor[HTML]{A2C4C9}0} & \multicolumn{1}{c|}{\cellcolor[HTML]{A2C4C9}1} & \multicolumn{1}{c|}{\cellcolor[HTML]{A2C4C9}1} & \multicolumn{1}{c|}{\cellcolor[HTML]{A2C4C9}0} & \multicolumn{1}{c|}{\cellcolor[HTML]{A2C4C9}1} & \multicolumn{1}{c|}{\cellcolor[HTML]{A2C4C9}1} & \multicolumn{1}{c|}{\cellcolor[HTML]{A2C4C9}0} & \multicolumn{1}{c|}{\cellcolor[HTML]{A2C4C9}0} & \multicolumn{1}{c|}{\cellcolor[HTML]{A2C4C9}0} & \multicolumn{1}{c|}{\cellcolor[HTML]{A2C4C9}0} & \multicolumn{1}{c|}{\cellcolor[HTML]{A2C4C9}0} & \multicolumn{1}{c|}{\cellcolor[HTML]{A2C4C9}0} & \multicolumn{1}{c|}{\cellcolor[HTML]{A2C4C9}0} & \multicolumn{1}{c|}{\cellcolor[HTML]{A2C4C9}0} & \multicolumn{1}{c|}{\cellcolor[HTML]{A2C4C9}0} & 0 \\ \hline
\rowcolor[HTML]{A4C2F4}
\multicolumn{1}{|c|}{\cellcolor[HTML]{A4C2F4}0} & \multicolumn{1}{c|}{\cellcolor[HTML]{A4C2F4}1} & 1 & \multicolumn{1}{c|}{\cellcolor[HTML]{A4C2F4}0} & \multicolumn{1}{c|}{\cellcolor[HTML]{A4C2F4}0} & \multicolumn{1}{c|}{\cellcolor[HTML]{A4C2F4}0} & \multicolumn{1}{c|}{\cellcolor[HTML]{A4C2F4}1} & \multicolumn{1}{c|}{\cellcolor[HTML]{A4C2F4}0} & \multicolumn{1}{c|}{\cellcolor[HTML]{A4C2F4}1} & \multicolumn{1}{c|}{\cellcolor[HTML]{A4C2F4}1} & \multicolumn{1}{c|}{\cellcolor[HTML]{A4C2F4}1} & \multicolumn{1}{c|}{\cellcolor[HTML]{A4C2F4}0} & \multicolumn{1}{c|}{\cellcolor[HTML]{A4C2F4}0} & \multicolumn{1}{c|}{\cellcolor[HTML]{A4C2F4}1} & \multicolumn{1}{c|}{\cellcolor[HTML]{A4C2F4}0} & \multicolumn{1}{c|}{\cellcolor[HTML]{A4C2F4}0} & \multicolumn{1}{c|}{\cellcolor[HTML]{A4C2F4}0} & \multicolumn{1}{c|}{\cellcolor[HTML]{A4C2F4}0} & \multicolumn{1}{c|}{\cellcolor[HTML]{A4C2F4}1} & \multicolumn{1}{c|}{\cellcolor[HTML]{A4C2F4}0} & \multicolumn{1}{c|}{\cellcolor[HTML]{A4C2F4}0} & 0 \\ \hline
\rowcolor[HTML]{9FC5E8}
\multicolumn{1}{|c|}{\cellcolor[HTML]{9FC5E8}1} & \multicolumn{1}{c|}{\cellcolor[HTML]{9FC5E8}0} & 0 & \multicolumn{1}{c|}{\cellcolor[HTML]{9FC5E8}0} & \multicolumn{1}{c|}{\cellcolor[HTML]{9FC5E8}1} & \multicolumn{1}{c|}{\cellcolor[HTML]{9FC5E8}1} & \multicolumn{1}{c|}{\cellcolor[HTML]{9FC5E8}0} & \multicolumn{1}{c|}{\cellcolor[HTML]{9FC5E8}0} & \multicolumn{1}{c|}{\cellcolor[HTML]{9FC5E8}1} & \multicolumn{1}{c|}{\cellcolor[HTML]{9FC5E8}0} & \multicolumn{1}{c|}{\cellcolor[HTML]{9FC5E8}1} & \multicolumn{1}{c|}{\cellcolor[HTML]{9FC5E8}0} & \multicolumn{1}{c|}{\cellcolor[HTML]{9FC5E8}1} & \multicolumn{1}{c|}{\cellcolor[HTML]{9FC5E8}0} & \multicolumn{1}{c|}{\cellcolor[HTML]{9FC5E8}0} & \multicolumn{1}{c|}{\cellcolor[HTML]{9FC5E8}0} & \multicolumn{1}{c|}{\cellcolor[HTML]{9FC5E8}0} & \multicolumn{1}{c|}{\cellcolor[HTML]{9FC5E8}1} & \multicolumn{1}{c|}{\cellcolor[HTML]{9FC5E8}0} & \multicolumn{1}{c|}{\cellcolor[HTML]{9FC5E8}0} & \multicolumn{1}{c|}{\cellcolor[HTML]{9FC5E8}0} & 0 \\ \hline
\rowcolor[HTML]{B4A7D6}
\multicolumn{1}{|c|}{\cellcolor[HTML]{B4A7D6}1} & \multicolumn{1}{c|}{\cellcolor[HTML]{B4A7D6}0} & 1 & \multicolumn{1}{c|}{\cellcolor[HTML]{B4A7D6}1} & \multicolumn{1}{c|}{\cellcolor[HTML]{B4A7D6}0} & \multicolumn{1}{c|}{\cellcolor[HTML]{B4A7D6}0} & \multicolumn{1}{c|}{\cellcolor[HTML]{B4A7D6}0} & \multicolumn{1}{c|}{\cellcolor[HTML]{B4A7D6}1} & \multicolumn{1}{c|}{\cellcolor[HTML]{B4A7D6}1} & \multicolumn{1}{c|}{\cellcolor[HTML]{B4A7D6}0} & \multicolumn{1}{c|}{\cellcolor[HTML]{B4A7D6}1} & \multicolumn{1}{c|}{\cellcolor[HTML]{B4A7D6}1} & \multicolumn{1}{c|}{\cellcolor[HTML]{B4A7D6}0} & \multicolumn{1}{c|}{\cellcolor[HTML]{B4A7D6}0} & \multicolumn{1}{c|}{\cellcolor[HTML]{B4A7D6}0} & \multicolumn{1}{c|}{\cellcolor[HTML]{B4A7D6}0} & \multicolumn{1}{c|}{\cellcolor[HTML]{B4A7D6}0} & \multicolumn{1}{c|}{\cellcolor[HTML]{B4A7D6}0} & \multicolumn{1}{c|}{\cellcolor[HTML]{B4A7D6}0} & \multicolumn{1}{c|}{\cellcolor[HTML]{B4A7D6}0} & \multicolumn{1}{c|}{\cellcolor[HTML]{B4A7D6}0} & 0 \\ \hline
\rowcolor[HTML]{D5A6BD}
\multicolumn{1}{|c|}{\cellcolor[HTML]{D5A6BD}1} & \multicolumn{1}{c|}{\cellcolor[HTML]{D5A6BD}1} & 0 & \multicolumn{1}{c|}{\cellcolor[HTML]{D5A6BD}0} & \multicolumn{1}{c|}{\cellcolor[HTML]{D5A6BD}0} & \multicolumn{1}{c|}{\cellcolor[HTML]{D5A6BD}1} & \multicolumn{1}{c|}{\cellcolor[HTML]{D5A6BD}1} & \multicolumn{1}{c|}{\cellcolor[HTML]{D5A6BD}0} & \multicolumn{1}{c|}{\cellcolor[HTML]{D5A6BD}1} & \multicolumn{1}{c|}{\cellcolor[HTML]{D5A6BD}0} & \multicolumn{1}{c|}{\cellcolor[HTML]{D5A6BD}1} & \multicolumn{1}{c|}{\cellcolor[HTML]{D5A6BD}0} & \multicolumn{1}{c|}{\cellcolor[HTML]{D5A6BD}0} & \multicolumn{1}{c|}{\cellcolor[HTML]{D5A6BD}0} & \multicolumn{1}{c|}{\cellcolor[HTML]{D5A6BD}0} & \multicolumn{1}{c|}{\cellcolor[HTML]{D5A6BD}1} & \multicolumn{1}{c|}{\cellcolor[HTML]{D5A6BD}0} & \multicolumn{1}{c|}{\cellcolor[HTML]{D5A6BD}1} & \multicolumn{1}{c|}{\cellcolor[HTML]{D5A6BD}0} & \multicolumn{1}{c|}{\cellcolor[HTML]{D5A6BD}0} & \multicolumn{1}{c|}{\cellcolor[HTML]{D5A6BD}1} & 0 \\ \hline
\end{longtable}

\end{center}
Con los valores obtenidos, implementamos la memoria en Quartus con la
programación VHDL obteniendo el siguiente código:
\section{Conclusiones}
\label{sec:orgdab2190}

\subsection*{Monsalvo Bolaños Melissa Monserrat}\label{sec:mons-bolan-melissa}

\subsection*{Romero Andrade Cristian}
\label{sec:romero-andr-crist}

\subsection*{Conclusiones en equipo}
\label{sec:concl-en-equipo}

\nocite{*}
\addcontentsline{toc}{section}{Referencias}
\printbibliography{}
\end{document}
